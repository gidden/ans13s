
%%%%%%%%%%%%%%%%%%%%%%%%%%%%%%%%%%%
\documentclass{anstrans}
\bibliographystyle{ans}

%%%%%%%%%%%%%%%%%%%%%%%%%%%%%%%%%%%
\usepackage{times} % fancier looking type
\usepackage{graphicx}
\usepackage{microtype} % if using PDF
\usepackage{amsmath} % for optimization equations
\usepackage{booktabs}
\usepackage{authblk}

%%%%%%%%%%%%%%%%%%%%%%%%%%%%%%%%%%%
\title{Developing Standardized, Open Benchmarks and a Corresponding 
Specification Language for the Simulation of Dynamic Fuel Cycles}
\author[*]{Matthew Gidden}
\author[$\dag$]{Anthony Scopatz}
\author[*]{Paul Wilson}
\email{gidden@wisc.edu}
\affil[*]{Department of Nuclear Engineering \& Engineering Physics, 
University of Wisconsin - Madison, Madison, WI, 53703}
\affil[$\dag$]{The Flash Center for Computational Science, University 
of Chicago, Chicago, IL, 60637}
\date{2012/06/29}

%%%%%%%%%%%%%%%%%%%%%%%%%%%%%%%%%%%
\begin{document}

%%%%%%%%%%%%%%%%%%%%%%%%%%%%%%%%%%%%%%%%%%%%%%%%%%%%%%%%%%%%%%%%%%%%%%
\section{Introduction}
Nuclear fuel cycle simulation is a field and practice that has involved a
variety of players in the nuclear fuel services arena, most predominately
governments and international organizations.  Additionally, multiple solution
frameworks have been proposed and implemented, including systems dynamics,
agent-based modeling, object oriented programming, and hybrids
thereof. Accordingly, there is strong motivation for consistency of testing
basic functionality as a means of verification and validation of the variety of
fuel cycle simulation applications. A number of such exercises have been
performed to date with a range of fidelity of scenario specification. This work
proposes a common language by which such specifications can be described and
provides some simple examples of its use.

\section{Overview}

%%%%%%%%%%%%%%%%%%%%%%%%%%%%%%%%%%%%%%%%%%%%%%%%%%%%%%%%%%%%%%%%%%%%%%
\subsection{Current Benchmark Landscape}
There have been a number of different benchmarking exercises attempted by
governmental and international agencies including the International Atomic
Energy Agency (IAEA) \cite{_international_2011} and the Nuclear Waste Technology
Review Board (NWTRB) \cite{_nuclear_2011}.  Both examples formulated their
benchmark specifications at meetings involving all participants, but have not
provided the full specifications to the public realm. MIT coordinated a
benchmarking exercise for dynamic simulators of advanced fuel cycles (i.e. fast
reactors with different conversion ratios) \cite{guerin_benchmark_2009},
including a specification with higher fidelity than those above. The
Organization for Economic Cooperation and Development (OECD) has also led a
benchmarking exercise \cite{boucher_benchmark_2012} and provided a more detailed
specification \cite{boucher_specification_2008}. 

%%%%%%%%%%%%%%%%%%%%%%%%%%%%%%%%%%%%%%%%%%%%%%%%%%%%%%%%%%%%%%%%%%%%%%
\subsection{Motivation}
Ubiquitous amongst all of the above-mentioned benchmarks is a lack of a common
language to describe scenario specifications. In general, specifications are
agreed upon during some number of meetings and/or conference calls and perhaps
are even iterated upon in order to match specific tunings required by some
simulators. Once set, a given specification is generally included as prose and
numeric tables in a final report. It should be noted that not all benchmarking
exercises provide fully formed scenarios, i.e., there are ``free parameters''
\cite{scopatz_fuel_2011}.

Developing a specification language would greatly benefit the fuel cycle
simulation community by allowing for a common, well-defined specification of a
set of scenario parameters to be discussed. Moreover, a computer-readable
scenario definition implementation would assist in the automation of benchmark
analyses. This is not an uncommon problem in the computational nuclear
community, and we look to the nuclear data community for inspiration. There has
been a push to update the Evaluated Nuclear Data Format (ENDF) utilizing a more
general language termed Generalized Nuclear Data (GND)
\cite{mattoon_generalized_2012}. This language is a \emph{specification} rather
than an \emph{implementation}, allowing others to tailor implementation details
(e.g. via XML, HDF5, etc.) to their needs. We propose using a similar formalism
to define scenarios that is independent of the analysis mechanism (e.g. system
dynamics, agent-based modeling).

Another mutual thread amongst previous benchmarking exercises is a shallow
treatment of the once-through fuel cycle. Although conceptually simple, a
thorough suite of benchmarks that cover both the facility-level and
simulation-wide metrics will provide insight into how the mechanics of various
simulators can affect benchmarking outcomes. Additionally, the once-through fuel
cycle is the basis from which any other fuel cycles will be expanded.
Accordingly, such a suite of benchmarks should be satisfied by any simulator
looking to analyze more advanced fuel cycles. This work provides a set of
suggested benchmarks written in the proposed specification language and
implemented in a computer-readable format.

%%%%%%%%%%%%%%%%%%%%%%%%%%%%%%%%%%%%%%%%%%%%%%%%%%%%%%%%%%%%%%%%%%%%%%
\subsection{Output Metrics}
Any benchmark must specify the output that is expected from the simulation. For
dynamic fuel cycles, the comparable output is myriad, and not all simulators are
able to provide all possible types of output. Examples of such output are:
electrical capacity, natural uranium consumption, SWU usage, amount of fuel
entering reactors, amount of fuel exiting reactors, amount of fuel in storage,
and the amount of separated material
\cite{boucher_specification_2008}\cite{guerin_benchmark_2009}. Each of the above
metrics is given on a per-year basis, which has historically been the
agreed-upon metric time step for comparison. Many are separated into
subcategories by facility type (e.g. types of reactors for electrical capacity)
or by element type (e.g. transuranics/plutonium/minor actinides in storage or
that have been separated). Furthermore, metrics related to material flow can be
broken down into their isotopic composition for comparison. This allows
additional metrics to be used, including nonproliferation metrics such as
plutonium heating at various locations in the fuel cycle. In general, it should
be expected that higher-level (i.e. elemental quantity) metrics will be easier
to match than lower-level metrics, for higher-level metrics correspond to
simulation mechanics fidelity whereas lower-level metrics correspond to physical
modeling fidelity.

For this work, the authors provide analytically derived metrics for very simple
(e.g. one reactor) cases. For larger cases of the once-through fuel cycle, the
simulation community can generally provide a consensus answer, and the authors
provide their solution for important high-level metrics such as natural uranium
and SWU consumption as well as mass flows exiting reactors. The metrics required
for advanced fuel cycles, i.e. fuel cycles with reprocessed fuel use, are more
difficult and the community has not agreed upon a ``right answer'' in prior
studies for most metrics except in simple cases with no growth after tuning
between simulators \cite{guerin_benchmark_2009}. This is an area that requires
continued collaboration and is prime for future work in the benchmarking arena.

%%
% leaving off here for now
%%
%%%%%%%%%%%%%%%%%%%%%%%%%%%%%%%%%%%%%%%%%%%%%%%%%%%%%%%%%%%%%%%%%%%%%%
\section{Scenario Specification Language}
Any benchmark must be defined in such a way that can be interpreted by different
simulators. Thus, an underlying structure exists that can be leveraged in order
to construct a common language by which different scenarios can be expressed. As
we progress in our work to classify and analyze benchmarks for the once-through
fuel cycle, we simultaneously construct a common language that define the
benchmarks. Language constructs include sets of parameters by which facilities
can be defined (e.g. the power level and cycle length of a reactor) as well as
generic containers for data series (e.g. defining a power demand curve through
piecewise functions or interpolated data points) much in the spirit of
\cite{mattoon_generalized_2012}. In addition to the language specification, an
example implementation will be provided.

%%%%%%%%%%%%%%%%%%%%%%%%%%%%%%%%%%%%%%%%%%%%%%%%%%%%%%%%%%%%%%%%%%%%%%
\section{Conclusion}
The fuel cycle simulation community has long been a dispersed group of actors
with a common goal of verification and validation of their analysis
tools. Providing a common suite of open benchmarks using a common scenario
definition language will facilitate the efficient pursuit of such an
endeavor. This work provides a first look at such tools and their implementation
for the once-through fuel cycle. A natural next step is to expand these tools
for advanced fuel cycles, perhaps first those with single MOX recycle and then
more generally for fuel recycle via breeder/burner reactors.

%%%%%%%%%%%%%%%%%%%%%%%%%%%%%%%%%%%%%%%%%%%%%%%%%%%%%%%%%%%%%%%%%%%%%%
\section{Acknowledgments}
This research is being performed using funding received from the DOE Office of
Nuclear Energy's Nuclear Energy University Programs.  The author thanks the NEUP
for its generous support.\\ 
\includegraphics[width=1.5in]{neup_logo_large.jpg}

%%%%%%%%%%%%%%%%%%%%%%%%%%%%%%%%%%%%%%%%%%%%%%%%%%%%%%%%%%%%%%%%%%%%%%
\bibliography{refs}
\end{document}
