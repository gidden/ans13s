
%%%%%%%%%%%%%%%%%%%%%%%%%%%%%%%%%%%
\documentclass{anstrans}
\bibliographystyle{ans}

%%%%%%%%%%%%%%%%%%%%%%%%%%%%%%%%%%%
\usepackage{times} % fancier looking type
\usepackage{graphicx}
\usepackage{microtype} % if using PDF
\usepackage{amsmath} % for optimization equations
\usepackage{booktabs}
\usepackage{authblk}

%%%%%%%%%%%%%%%%%%%%%%%%%%%%%%%%%%%
\title{Developing Standardized, Open Benchmarks for Simulation of the 
Once-Through Fuel Cycle}
\author[*]{Matthew Gidden}
\author[$\dag$]{Anthony Scopatz}
\author[*]{Paul Wilson}
\email{gidden@wisc.edu}
\affil[*]{Department of Nuclear Engineering \& Engineering Physics, 
University of Wisconsin - Madison, Madison, WI, 53703}
\affil[$\dag$]{The Flash Center for Computational Science, University 
of Chicago, Chicago, IL, 60637}
\date{2012/06/29}

%%%%%%%%%%%%%%%%%%%%%%%%%%%%%%%%%%%
\begin{document}

%%%%%%%%%%%%%%%%%%%%%%%%%%%%%%%%%%%%%%%%%%%%%%%%%%%%%%%%%%%%%%%%%%%%%%
\section{Introduction}
Nuclear fuel cycle simulation is a field and practice that has 
involved a variety of players in the nuclear fuel services arena. 
Additionally, multiple solution frameworks have been proposed and 
implemented, including systems dynamics and object oriented 
programming. 

There have been a number of different benchmarking exercises attempted
by governmental and international agencies including the Organization 
for Economic Cooperation and Development (OECD) \cite{_advanced_2006}

\section{Benchmark Classification}
The types of benchmarks that are applicable to fuel cycle simulation
come in two main flavors: facility boundary benchmarks, e.g. 

Hi \cite{cyclus2012}

%%%%%%%%%%%%%%%%%%%%%%%%%%%%%%%%%%%%%%%%%%%%%%%%%%%%%%%%%%%%%%%%%%%%%%
\section{Acknowledgments}
This research is being performed using funding received from the DOE
Office of Nuclear Energy's Nuclear Energy University Programs.  The
author thanks the NEUP for its generous support.

%%%%%%%%%%%%%%%%%%%%%%%%%%%%%%%%%%%%%%%%%%%%%%%%%%%%%%%%%%%%%%%%%%%%%%
\bibliography{refs}
\end{document}
