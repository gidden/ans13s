
%%%%%%%%%%%%%%%%%%%%%%%%%%%%%%%%%%%
\documentclass{anstrans}
\bibliographystyle{ans}

%%%%%%%%%%%%%%%%%%%%%%%%%%%%%%%%%%%
\usepackage{times} % fancier looking type
\usepackage{graphicx}
\usepackage{microtype} % if using PDF
\usepackage{amsmath} % for optimization equations
\usepackage{booktabs}
\usepackage{authblk}
\usepackage{moreverb} % for verbatim snippets of code
\usepackage{fancyvrb}

%%%%%%%%%%%%%%%%%%%%%%%%%%%%%%%%%%%
\title{Developing Standardized, Open Benchmarks and a Corresponding 
Specification Language for the Simulation of Dynamic Fuel Cycles}
\author[*]{Matthew Gidden}
\author[$\dag$]{Anthony Scopatz}
\author[*]{Paul Wilson}
\email{gidden@wisc.edu}
\affil[*]{Department of Nuclear Engineering \& Engineering Physics, 
University of Wisconsin - Madison, Madison, WI, 53703}
\affil[$\dag$]{The Flash Center for Computational Science, University 
of Chicago, Chicago, IL, 60637}
\date{2012/06/29}

%%%%%%%%%%%%%%%%%%%%%%%%%%%%%%%%%%%
\begin{document}

%%%%%%%%%%%%%%%%%%%%%%%%%%%%%%%%%%%%%%%%%%%%%%%%%%%%%%%%%%%%%%%%%%%%%%
\section{Introduction}
Nuclear fuel cycle simulation is a field and practice that has involved a
variety of players in the nuclear fuel services arena, most predominately
governments and international organizations.  Additionally, multiple solution
frameworks have been proposed and implemented, including systems dynamics,
agent-based modeling, object oriented programming, and hybrids
thereof. Accordingly, there is strong motivation for consistency of testing
basic functionality as a means of verification and validation of the variety of
fuel cycle simulation applications. A number of such exercises have been
performed to date with a range of fidelity of scenario specification. This work
proposes a common language by which such specifications can be described and
some simple benchmarks missing from the current literature that can provide an
examples of its use.

%%%%%%%%%%%%%%%%%%%%%%%%%%%%%%%%%%%%%%%%%%%%%%%%%%%%%%%%%%%%%%%%%%%%%%
\section{Overview}

%%%%%%%%%%%%%%%%%%%%%%%%%%%%%%%%%%%%%%%%%%%%%%%%%%%%%%%%%%%%%%%%%%%%%%
\subsection{Current Benchmark Landscape}
There have been a number of different benchmarking exercises attempted by
governmental and international agencies including the International Atomic
Energy Agency (IAEA) \cite{_international_2011} and the Nuclear Waste Technology
Review Board (NWTRB) \cite{_nuclear_2011}.  Both examples formulated their
benchmark specifications at meetings involving all participants, but have not
provided the full specifications to the public realm. MIT coordinated a
benchmarking exercise for dynamic simulators of advanced fuel cycles (i.e. fast
reactors with different conversion ratios) \cite{guerin_benchmark_2009},
including a specification with higher fidelity than those above. The
Organization for Economic Cooperation and Development (OECD) has also led a
benchmarking exercise that covers an example each of open, modified open, and
closed fuel cycles \cite{boucher_benchmark_2012} and provided a more detailed
specification \cite{boucher_specification_2008}.

%%%%%%%%%%%%%%%%%%%%%%%%%%%%%%%%%%%%%%%%%%%%%%%%%%%%%%%%%%%%%%%%%%%%%%
\subsection{Motivation}
Ubiquitous amongst all of the above-mentioned benchmarks is a lack of a common
language to describe scenario specifications. In general, specifications are
agreed upon during some number of meetings and/or conference calls and perhaps
are even iterated upon in order to match specific tunings required by some
simulators. Once set, a given specification is generally included as prose and
numeric tables in a final report. It should be noted that not all benchmarking
exercises provide fully specified scenarios. In these cases there are free, 
unconstrained parameters which simulators that require such inputs must make
arbitrary best estimates for their values \cite{scopatz_fuel_2011}.

Developing a specification language would greatly benefit the fuel cycle
simulation community by allowing for a common and well-defined set of scenario
parameters to be discussed. Moreover, a computer-readable scenario definition
implementation would assist in the automation of benchmark analyses. This is not
an uncommon problem in computational nuclear engineering. We have thus looked to
the nuclear data community for inspiration.  There has been a push to update the
Evaluated Nuclear Data Format (ENDF) utilizing a more general language termed
Generalized Nuclear Data (GND) \cite{mattoon_generalized_2012}. This language is
a \emph{specification} rather than an \emph{implementation}, allowing others to
tailor implementation details (e.g. via XML, HDF5, etc.) to their needs. We
propose using a similar formalism to define scenarios that is independent of the
analysis mechanism (e.g. system dynamics, agent-based modeling).

Another mutual thread amongst previous benchmarking exercises is a shallow
treatment of low-level benchmarks. Although conceptually simple, a thorough
suite of benchmarks that cover both the facility-level and simulation-wide
metrics will provide insight into how the mechanics of various simulators can
affect benchmarking outcomes. Again looking to previous works in computational
nuclear engineering, we propose small problems with analytical solutions similar
to other benchmarking exercises \cite{wagner_mcnp:_1992}. We add mostly to the
available set of once-through simulation benchmarks, as the once-through fuel
cycle is the basis from which any other fuel cycles will be expanded.
Accordingly, such a suite of benchmarks should be satisfied by any simulator
looking to analyze more advanced fuel cycles. This work provides a set of
suggested benchmarks written in the proposed specification language and
implemented in a computer-readable format.

%%%%%%%%%%%%%%%%%%%%%%%%%%%%%%%%%%%%%%%%%%%%%%%%%%%%%%%%%%%%%%%%%%%%%%
\subsection{Output Metrics}
Any benchmark must specify the output that is expected from the simulation. For
dynamic fuel cycles, the comparable output is myriad, and not all simulators are
able to provide all possible types of output. Examples of such output are:
electrical capacity, natural uranium consumption, SWU usage, amount of fuel
entering reactors, amount of fuel exiting reactors, amount of fuel in storage,
and the amount of separated material
\cite{boucher_specification_2008}\cite{guerin_benchmark_2009}. Each of the above
metrics is given on a per-year basis, which has historically been the
agreed-upon metric time step for comparison. Many are separated into
subcategories by facility type (e.g. types of reactors for electrical capacity)
or by element type (e.g. transuranics/plutonium/minor actinides in storage or
that have been separated). Furthermore, metrics related to material flow can be
broken down into their isotopic composition for comparison. This allows
additional metrics to be used, including nonproliferation metrics such as
plutonium heating at various locations in the fuel cycle. In general, it should
be expected that higher-level (i.e. elemental quantity) metrics will be easier
to match than lower-level metrics, for higher-level metrics correspond to
simulation mechanics fidelity whereas lower-level metrics correspond to physical
modeling fidelity.

For this work, the authors provide analytically derived metrics for very simple
(e.g. one reactor) cases. For larger cases of the once-through fuel cycle, the
simulation community can generally provide a consensus answer, and the authors
provide their solution for important high-level metrics such as natural uranium
and SWU consumption as well as mass flows exiting reactors. The metrics required
for advanced fuel cycles, i.e. fuel cycles with reprocessed fuel use, are more
difficult and the community has not agreed upon a ``right answer'' in prior
studies for most metrics except in simple cases with no growth after tuning
between simulators \cite{guerin_benchmark_2009}. This is an area that requires
continued collaboration and is prime for future work in the benchmarking arena.

%%%%%%%%%%%%%%%%%%%%%%%%%%%%%%%%%%%%%%%%%%%%%%%%%%%%%%%%%%%%%%%%%%%%%%
\section{Contributed Work}

%%%%%%%%%%%%%%%%%%%%%%%%%%%%%%%%%%%%%%%%%%%%%%%%%%%%%%%%%%%%%%%%%%%%%%
\subsection{Scenario Specification Language}
Any benchmark must be defined in such a way that it can be interpreted by
different simulators. There is, therefore, an underlying structure that can be
leveraged in order to construct a common language by which different scenarios
can be expressed. Such a language needs to minimally define simulation input as
to cover all required specifics. What is meant by ``minimally defining'' a
simulation is informed by the output metrics used in the benchmark. For example,
in prior benchmarking exercises, the number of reactors of a given type in
service is an agreed upon metric. Accordingly, information is required regarding
when reactors enter the simulation, i.e. a demand curve, and when reactors leave
the simulation, i.e. a lifetime. For this metric, licensing and construction
times are extraneous information vis-a-vis simulation mechanics -- that is to
say: if a simulator uses them, they inform a pre-planning process to meet the
required demand curve; the output metric is only related to the demand curve
being met at the requisite time.

Determining the minimal set of information to describe these benchmark scenarios
is a simulation-community-level discussion topic. The authors have proposed such
a set of information to describe the once-through fuel cycle as a
proof-of-principle, and look to future work and the community to continue to
develop the language for advanced cycle cases. 

In order to describe scenario definitions, there are largely three main
categories of parameters: materials used in the simulation, facilities used in
the simulation, and over-arching fuel-cycle scenario information (such as power
curves). We specify these parameters using language constructs such as
dictionaries, lists, and attributes. We include the latter construct in order to
encapsulate additional information about units and information that may be
specific to a subset of simulators but its not necessarily housed in the minimum
set of required parameters. An overview of the proposed specification
language and implementation examples in JavaScript Object Notation
(JSON)\cite{_json_2001} follows. All examples are for taken from for the NEA 1a
benchmark\cite{boucher_specification_2008}.

\subsubsection{Materials}
Materials in a fuel cycle simulation generally come in two flavors: specified
recipes, which are normally defined as input for reactors, and the resulting
output which is a function of the input and applied burnup. We denote the two
flavors as being recipes or not. To begin, all material types are specified by a
given name. At the current time, we apply a rigid definition of constraints on
recipes, i.e., there are specified concentration values for each isotope in the
recipe, and a density is provided. Because the output is a result of a physics
calculation which may be implemented with varying fidelity, a less rigid
approach is used for non-recipes. The specification does not explicitly name all
isotopes in the output fuel; rather, it defines actual constraints on certain
isotopes that are physically provable. Finally, we provide a parent field which
takes as an argument another material. This is a short-hand way of denoting
trees of materials, i.e., multiple materials with mutual constraints. An example
is provided in Figure~\ref{fig:material}.

\begin{figure}[h!]
\begin{Verbatim}[frame=single]
"materials": {
  "nea_leu": {
    "recipe": true,
    "attributes" : {
    "concentrations": ["float","atom"],
    "density": ["float","g/cc"]
   },
   "constraints": [        
     ["U235", 0.0495],
     ["U238", 0.9505],
     ["O16", 2.0],
     ["density", 10.2]
   ], 
   "parents": []
  },
  "nea_spent_pwr_uox": {
    "recipe": false,
    "attributes" : {
    "concentrations": ["float","atom"]
    }
    "constraints": [
      id == 92235 && x < 0.0495",
      "id == 92238 && x < 0.9505"
    ], 
    "parents": []
  }
}
\end{Verbatim}
\caption{NEA1a Materials Implemented in JSON}
\label{fig:material}
\end{figure}

\subsubsection{Facilities}
Facilities in the specification language have many of the same constructs as
materials. Each facility type has a unique name, an attribute list which
declares the units for parameters, and a constraint list which declares the
values for parameters. Facilities must also specify the types of input and
output materials with which they deal. For each type of facility, e.g. reactors,
separations facilities, etc., a specification is defined. An example for the
reactor in the NEA 1a benchmark is provided in Figure~\ref{fig:facility}.

\begin{figure}[h!]
\begin{Verbatim}[frame=single]
"facilities": {
  "lwr_reactor": {
    "type":"reactor",
    "attributes": {
      "thermal_power": ["float", "GWt"],
      "efficiency": ["float", "percent"],
      "burnup": ["float", "GWd/tHM"],
      "storage_time": ["int", "year"],
      "cooling_time": ["int", "year"],
      "cycle_length": ["int", "month"],
      "core_loading": ["float", "tHM"],
      "batch_number": "int",
      "lifetime": ["int", "year"]
    },
    "constraints": [
      ["thermal_power", 4.25],
      ["efficiency", 34.1],
      ["burnup", 60],
      ["storage_time", 2],
      ["cooling_time", 5],
      ["cycle_length", 12],
      ["core_loading", 78.7],
      ["nbatches", 3],
      ["lifetime", 60]
    ],
    "inputs": ["nea_leu"],
    "outputs": ["nea_spent_pwr_uox"]
  },
}
\end{Verbatim}
\caption{NEA1a Reactor Facility Implemented in JSON}
\label{fig:facility}
\end{figure}

\subsubsection{Fuel Cycle}
The final section of the scenario specification is that related to global fuel
cycle parameters, encapsulating the dynamism of the simulation. This section
describes the time scale of the simulation, the power demand for the system as a
function of reactor type, any initial facilities in the simulation, and a notion
of technology availability, i.e., time points at which facilities can be/are to
be built.

\subsubsection{Translators}
As a final note, the authors would like to point out that the specification
language is intended to be ubiquitous. Any benchmark fuel cycle scenario can be
written in the language, allowing the simulation community to be able to refer
to a single, unambiguous specification source. We fully expect there to be some
work taken on the part of the simulation developers to translate the
specification into a valid input for their simulator. As part of this work, we
have implemented such a translator in Python for the Cyclus fuel cycle
simulator \cite{cyclus2012}.

%%%%%%%%%%%%%%%%%%%%%%%%%%%%%%%%%%%%%%%%%%%%%%%%%%%%%%%%%%%%%%%%%%%%%%
\subsection{Once-Through Benchmarks}
The goal of the once-through verification benchmark suite is to provide a
low-level view of simulation mechanics in order to gain insight into differences
amongst simulators in more advanced benchmarks. We focus this suite on two main
areas: the entrance and exit of reactors in the simulation as well as material
flows into and out of reactors. Input flows are analyzed in terms of quanitity,
SWU usage, and natural uranium usage. Output flows are analyzed in terms of
quantity.

We focus on two classes of benchmarks, those involving one type of reactor
(PWRs) and those involving two (PWRs and HWRs). To investigate the static
building mechanisms and simplest material flows, the first benchmark set will
last for one reactor lifetime. The second set will investigate simple dynamic
reactor building, utilizing a simulation with exponential growth curves for
power demand. For the final set we will describe and analyze the low-demand
once-through case give in the INPRO benchmark \cite{_international_2011}.

%%%%%%%%%%%%%%%%%%%%%%%%%%%%%%%%%%%%%%%%%%%%%%%%%%%%%%%%%%%%%%%%%%%%%%
\subsection{Recycle Benchmarks}
The modeling of fuel cycles that utilize recycling of spent nuclear fuel
introduces a number of complexities, including the need to model supporting
facilities. As with the once-through benchmarks, we propose a series of
benchmarks increasing in simulation size, adding complexity along the way. The
end goal is a scenario benchmark that captures the set of facets involved in
modeling recycling cases. Having isolated any simulator mechanics issued related
to the introduction of facilities in the once-through benchmarking case, we
focus the recycle benchmarks on the interaction between advanced reactors and
their supporting facilities.

We again focus on two classes of benchmarks following the classes of fuel cycles
that use recycling: modified open, where recycling involves a single stream of
material, and full recycle, where recycling involves more than one stream of
material. This is a similar strategy to previous benchmarking exercises
\cite{boucher_specification_2008}. We, however, propose a first scenario that
utilizes a single reactor fueled with recycled material in each case. This lays
the foundation for comparisons amongst simulators for the 1-reactor case much
like in the once-through benchmark suite. The next benchmark scenario in each
class calls for multiple reactors to be built (again without growth in demand
for the advanced reactors). The final benchmark models an expansion of advanced
reactors in order to capture the symbiotic relationship between the reactors
that use fresh fuel and those that use recycled fuel. 

%%%%%%%%%%%%%%%%%%%%%%%%%%%%%%%%%%%%%%%%%%%%%%%%%%%%%%%%%%%%%%%%%%%%%%
\section{Conclusion}
The fuel cycle simulation community has long been a dispersed group of actors
with a common goal of verification and validation of their analysis
tools. Providing a common suite of open benchmarks using a common scenario
definition language will facilitate the efficient pursuit of such an
endeavor. This work provides a first look at such tools and their implementation
for the once-through and recycle fuel cycles. The once-through case is important
because of it's simplicity - it allows for direct, simple comparisons among
various simulation mechanics regarding building decisions and material
flows. The goal of this work is to provide the community with an open set of
fully specified benchmarks that range from the very simple, in order to measure
the difference in simulation mechanics between simulators, to the more
complicated, designed to highlight structural differences in the way simulators
deal with full-fledged scenarios that include material recycling. From this
basis, the community can make meaningful comparisons amongst simulators and grow
the set of available benchmarks in a common language.

%%%%%%%%%%%%%%%%%%%%%%%%%%%%%%%%%%%%%%%%%%%%%%%%%%%%%%%%%%%%%%%%%%%%%%
\section{Acknowledgments}
This research is being performed using funding received from the DOE Office of
Nuclear Energy's Nuclear Energy University Programs.  The author thanks the NEUP
for its generous support.\\ 
\includegraphics[width=1.5in]{neup_logo_large.jpg}

%%%%%%%%%%%%%%%%%%%%%%%%%%%%%%%%%%%%%%%%%%%%%%%%%%%%%%%%%%%%%%%%%%%%%%
\bibliography{refs}
\end{document}
