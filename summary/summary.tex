
%%%%%%%%%%%%%%%%%%%%%%%%%%%%%%%%%%%
\documentclass{anstrans}
\bibliographystyle{ans}

%%%%%%%%%%%%%%%%%%%%%%%%%%%%%%%%%%%
\usepackage{times} % fancier looking type
\usepackage{graphicx}
\usepackage{microtype} % if using PDF
\usepackage{amsmath} % for optimization equations
\usepackage{booktabs}
\usepackage{authblk}

%%%%%%%%%%%%%%%%%%%%%%%%%%%%%%%%%%%
\title{Developing Standardized, Open Benchmarks and Scenario 
Definitions for Simulation of the Once-Through Fuel Cycle}
\author[*]{Matthew Gidden}
\author[$\dag$]{Anthony Scopatz}
\author[*]{Paul Wilson}
\email{gidden@wisc.edu}
\affil[*]{Department of Nuclear Engineering \& Engineering Physics, 
University of Wisconsin - Madison, Madison, WI, 53703}
\affil[$\dag$]{The Flash Center for Computational Science, University 
of Chicago, Chicago, IL, 60637}
\date{2012/06/29}

%%%%%%%%%%%%%%%%%%%%%%%%%%%%%%%%%%%
\begin{document}

%%%%%%%%%%%%%%%%%%%%%%%%%%%%%%%%%%%%%%%%%%%%%%%%%%%%%%%%%%%%%%%%%%%%%%
\section{Introduction}
Nuclear fuel cycle simulation is a field and practice that has 
involved a variety of players in the nuclear fuel services arena, most
predominately governments and international organizations. 
Additionally, multiple solution frameworks have been proposed and 
implemented, including systems dynamics, agent-based modeling, object 
oriented programming, and hybrids thereof. Accordingly, there is 
strong motivation for consistency of testing basic functionality as a
means of verification and validation of the variety of fuel cycle
simulation applications.

%%%%%%%%%%%%%%%%%%%%%%%%%%%%%%%%%%%%%%%%%%%%%%%%%%%%%%%%%%%%%%%%%%%%%%
\section{Current Benchmark Landscape}
There have been a number of different benchmarking exercises attempted
by governmental and international agencies including the International
Atomic Energy Agency (IAEA) \cite{_international_2011} and the
Nuclear Waste Technology Review Board (NWTRB) \cite{_nuclear_2011}.
Both examples formulated their benchmark specifications at meetings
involving all participants, but have not provided the full 
specifications to the public realm. The Organization for Economic 
Cooperation and Development (OECD) has also led a benchmarking 
exercise \cite{boucher_benchmark_2012} and provided a more detailed
specification \cite{boucher_specification_2008}.

%%%%%%%%%%%%%%%%%%%%%%%%%%%%%%%%%%%%%%%%%%%%%%%%%%%%%%%%%%%%%%%%%%%%%%
\section{Motivation}
Common amongst all of the above-mentioned benchmarks is a shallow
treatment of the once-through fuel cycle. Although conceptually 
rather simple, the community will benefit from a thorough suite of 
benchmarks that cover both the facility-level and simulation-wide 
metrics. Additionally, it is the basis from which any other fuel 
cycles will be expanded, thus such a suite of benchmarks should be 
satisfied by any simulator looking to analyze more advanced fuel 
cycles.

Another mutual thread amongst previous benchmarking exercises is a 
lack of a common language to describe scenario specifications. Such 
an advance would greatly benefit the fuel cycle simulation community 
by allowing for a common, well-defined set of scenario parameters to 
be discussed. Moreover, a computer-readable scenario definition 
specification would assist in the automation of benchmark
analyses. This is not an uncommon problem in the computational 
nuclear community, and we look to the nuclear data community for 
inspiration. There has been a push to update the Evaluated Nuclear
Data Format (ENDF) utilizing a more general language termed 
Generalized Nuclear Data (GND) \cite{mattoon_generalized_2012}. This
language is a \emph{specification} rather than an 
\emph{implementation}, allowing others to tailor implementation 
details (e.g. via XML, HDF5, etc.) to their needs. We propose using a 
similar formalism to define scenarios that is independent of the 
analysis mechanism (e.g. system dynamics, agent-based modeling).

%%%%%%%%%%%%%%%%%%%%%%%%%%%%%%%%%%%%%%%%%%%%%%%%%%%%%%%%%%%%%%%%%%%%%%
\section{Benchmark Classification}
The types of benchmarks that are applicable to fuel cycle simulation
come in two main flavors: facility boundary benchmarks, e.g. the
isotopic composition of fuel exiting a reactor given input composition
and reactor facility parameters, simulation-wide benchmarks, e.g. the
amount of spent fuel exiting reactors given a specified demand curve.
Fully defining benchmarks requires agreed-upon metrics (i.e. output).
Previous work notes that natural uranium consumption, SWU 
requirements, material flows between facility boundaries, and the 
isotopics of those flows as a function of time are chief metrics by
which to measure these types of benchmarks 
\cite{boucher_specification_2008}. This work will analyze and 
categorize the types of benchmarks applicable to the once-through 
fuel cycle and the metrics associated with each in as general a way
as possible in order to incorporate the different solution mechanisms
taken by the fuel cycle simulation community.

%%%%%%%%%%%%%%%%%%%%%%%%%%%%%%%%%%%%%%%%%%%%%%%%%%%%%%%%%%%%%%%%%%%%%%
\section{Scenario Specification Language}
Any benchmark must be defined in such a way that can be interpreted by
different simulators. Thus, an underlying structure exists that can be
leveraged in order to construct a common language by which different
scenarios can be expressed. As we progress in our work to classify and
analyze benchmarks for the once-through fuel cycle, we simultaneously
construct a common language that define the benchmarks. Language 
constructs include sets of parameters by which facilities can be
defined (e.g. the power level and cycle length of a reactor) as well
as generic containers for data series (e.g. defining a power demand 
curve through piecewise functions or interpolated data points) much
in the spirit of \cite{mattoon_generalized_2012}. In addition to the
language specification, an example implementation will be provided.

%%%%%%%%%%%%%%%%%%%%%%%%%%%%%%%%%%%%%%%%%%%%%%%%%%%%%%%%%%%%%%%%%%%%%%
\section{Conclusion}
The fuel cycle simulation community has long been a dispersed group
of actors with a common goal of verification and validation of their
analysis tools. Providing a common suite of benchmarks and a common
scenario definition language is needed in order to efficiently pursue
such an endeavor. This works provides a first look at such tools for
the once-through fuel cycle. A natural next step is to expand these
tools for advanced fuel cycles, perhaps first those with single MOX
recycle and then more generally for fuel recycle via breeder/burner
reactors.

%%%%%%%%%%%%%%%%%%%%%%%%%%%%%%%%%%%%%%%%%%%%%%%%%%%%%%%%%%%%%%%%%%%%%%
\section{Acknowledgments}
This research is being performed using funding received from the DOE
Office of Nuclear Energy's Nuclear Energy University Programs.  The
author thanks the NEUP for its generous support.

%%%%%%%%%%%%%%%%%%%%%%%%%%%%%%%%%%%%%%%%%%%%%%%%%%%%%%%%%%%%%%%%%%%%%%
\bibliography{refs}
\end{document}
