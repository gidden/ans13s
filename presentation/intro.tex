% Overview : intro.tex
% Talk about the current state of affairs

\begin{frame}[ctb!]
  \frametitle{Introduction : Fuel Cycle Simulation}
  Fuel cycle simulation (FCS) is designed to answer policy-related questions
  regarding transitions from one equilibrium state to another.

  \vspace{0.4cm}

  A simulator answers the following questions as a function of its 
  parameter space:
  \begin{itemize}
    \item how much material exists
    \item where does that material reside
    \item from/to where and when is material transported
    \item what kinds of facilities are needed
    \item when is each type of facility needed
  \end{itemize}
\end{frame}

\begin{frame}[ctb!]
  \frametitle{Introduction : Modeling Choices}
  Each simulation effort has a number of choices when modeling the nuclear fuel
  cycle, including:
  \begin{itemize}
    \item what simulation engine kernel to use
    \item what platform to build their simulation upon
    \item the level of complexity with which to model simulation entity
      interaction
    \item the level of fidelity with which to model physical processes
    \item the time-scale granularity
  \end{itemize}
\end{frame}

\begin{frame}[ctb!]
  \frametitle{Introduction : FCS Development}
  A number of different simulators are in active development around the world. 

  \begin{itemize}
    \item CAFCA - VENSIM \cite{busquim_e_silva_system_2008}
    \item COSI - Java \cite{boucher_cosi:_2006}
    \item Cyclus - C++ \cite{cyclus2012}
    \item DANESS - Stella/iThink \cite{van_den_durpel_daness_2009}
    \item NUWASTE - Excel \cite{_nuclear_2011}
    \item VISION - Powersim \cite{jacobson_verifiable_2010}
    \item DESAE, FAMILY, VISTA, etc.
  \end{itemize}
\end{frame}
