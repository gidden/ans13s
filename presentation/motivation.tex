% Overview : motivation.tex
% Explain why this talk is being given.

\begin{frame}[ctb!]
  \frametitle{Motivation : Benchmarks Needed} 
  With so many players implementing simulations via various methods, we need 
  validation and verification (i.e., benchmarks)!

  \vspace{0.4cm}

  First things first: what's makes up a benchmark for FCSs?

  \begin{itemize}
    \item a scenario
    \item a set of facilities (and their connections)
    \item a set of materials
  \end{itemize}
\end{frame}

\begin{frame}[ctb!]
  \frametitle{Motivation : Previous Exercises}
  A number of exercises have been initiated by the FCS community.

  \vspace{0.4cm}

  Just to name a few:
  \begin{itemize}
    \item IAEA/INPRO (15 participants [INPRO nation states])\cite{_international_2011}
    \item MIT (4 participants) \cite{guerin_benchmark_2009}
    \item NUWASTE (5 participants) \cite{abkowitz_workshop_2011}
    \item OECD/NEA (5 participants) \cite{boucher_benchmark_2012}
  \end{itemize}
\end{frame}

\begin{frame}[ctb!]
  \frametitle{Motivation : How Benchmarks are Formed}
  A ``base case'' is considered as the starting point.

  \vspace{0.4cm}

  Some (small) number of advanced fuel cycles are considered as additional cases.

  \vspace{0.4cm}

  An iterative, communal process produces scenario, facility, and material
  parameters.

  \vspace{0.4cm}

  The agreed-upon parameters are shared among the participants.

  \vspace{0.4cm}

  Results are presented at follow-up meetings or in publications in graph form.
\end{frame}

\begin{frame}[ctb!]
  \frametitle{Motivation : Reproducibility}
  FCS benchmarks are a V\&V exercise. They require hard data and complete 
  information.

  \vspace{0.4cm}

  Many benchmarks specify a fuel cycle \textit{template}, but do not fully 
  specify the required parameters of the fuel cycle.

  \vspace{0.4cm}

  Critically: correctness can be (minimally) confirmed by matching an under
  specified benchmark, but incorrectness can not be confirmed by failing to
  match an under specified benchmark.
\end{frame}

\begin{frame}[ctb!]
  \frametitle{Motivation : Moving V\&V Exercises Forward}
  We can learn from other nuclear computational science communities.
  \begin{itemize}
    \item nuclear data \cite{mattoon_generalized_2012}
    \item nuclear transport / criticality
  \end{itemize}

  An ideal solution is a community-consensus standard way to describe benchmarks.

  \vspace{0.4cm}

  Fuel cycles are difficult to describe. We need a common language or structure
  to talk about what exactly a fuel cycle is.

  \vspace{0.4cm}

  Having a better way to discuss/describe a fuel cycle allows us to fill in
  missing holes in any specified benchmark.
\end{frame}

\begin{frame}[ctb!]
  \frametitle{Motivation : Goals of FCS V\&V Evolution}
  What goals should a valid solution strive towards?
  \begin{enumerate}
    \item A solution should be complete; it should be valid for any current or
      future participant
    \item A solution should be \textit{easy} to work with, either through
      automation or by-hand comprehension.
    \item A solution should be \textit{public and open}, both regarding input
      and output data.
  \end{enumerate}

  Any proposal needs buy-in from the FCS, specifically regarding:
  \begin{enumerate}
    \item what parameters belong in a \textit{full} fuel cycle description
    \item what are valid metrics to benchmark against
    \item how should these metrics be aggregated (e.g., by month? by year?; by
      element? by isotope?)
  \end{enumerate}
\end{frame}
