% Overview : motivation.tex
% Explain why this talk is being given.

\begin{frame}[ctb!]
  \frametitle{Motivation : Benchmarks Needed} 
  With so many players implementing simulations via various methods, we need 
  validation and verification (i.e., benchmarks)!

  First things first: what's makes up a benchmark for FCSs?

  \begin{itemize}
    \item a scenario
    \item a set of facilities (and their connections)
    \item a set of materials
  \end{itemize}
\end{frame}

\begin{frame}[ctb!]
  \frametitle{Motivation : Previous Exercises}
  A number of excercises have been initiated by the FCS community.

  Just to name a few:
  \begin{itemize}
    \item IAEA/INPRO (15 participants [INPRO nation states])\cite{_international_2011}
    \item MIT (4 participants) \cite{guerin_benchmark_2009}
    \item NUWASTE (5 participants) \cite{abkowitz_workshop_2011}
    \item OECD/NEA (5 participants) \cite{boucher_benchmark_2012}
  \end{itemize}
\end{frame}

\begin{frame}[ctb!]
  \frametitle{Motivation : How Benchmarks are Formed}
  A ``base case'' is considered as the starting point.

  Some (small) number of advanced fuel cycles are considered as addtional cases.

  An iterative, communal process produces scenario, facility, and material
  parameters.

  The agreed-upon parameters are shared amongst the participants.

  Results are presented at follow-up meetings or in publications in graph form.
\end{frame}

\begin{frame}[ctb!]
  \frametitle{Motivation : Reproducibility}
  FCS benchmarks are a V\&V exercise. They require hard data and complete 
  information.

  Many benchmarks specify a fuel cycle \textit{template}, but do not fully 
  specify the required parameters of the fuel cycle.

  Critically: correctess can be (minimally) confirmed by matching an under
  specified benchmark, but incorrectness can not be confirmed by failing to
  match an under specified benchmark.
\end{frame}
